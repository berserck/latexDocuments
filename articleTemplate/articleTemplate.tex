\documentclass[12pt, a4paper,twoside]{article}
\usepackage{geometry}                % See geometry.pdf to learn the layout options. There are lots.
\geometry{a4paper}                   % ... or a4paper or a5paper or ... 
%\geometry{landscape}                % Activate for for rotated page geometry
%\usepackage[parfill]{parskip}    % Activate to begin paragraphs with an empty line rather than an indent

\usepackage{graphicx}
\usepackage{amssymb}
\usepackage{epstopdf}
\DeclareGraphicsRule{.tif}{png}{.png}{`convert #1 `dirname #1`/`basename #1 .tif`.png}

%use the fullpage package for somewhat standardized smaller margins
\usepackage[cm]{fullpage}

% for fancy section font and color
\usepackage[x11names]{xcolor}
\definecolor{section_color}{RGB}{00,66,99}             %section header color
\definecolor{subsection_color}{RGB}{72, 138, 199}% subsection header color
\definecolor{title_color}{RGB}{72, 138, 199}              % title color

%improving on the abstract (changing color and size)
\usepackage{abstract}
\renewcommand{\abstractnamefont}{\color{section_color}\fontfamily{phv}\selectfont\LARGE}
%\renewcommand{\abstracttextfont}{\normalfont}

% fancy sections
\usepackage{helvet}
\usepackage{sectsty}
\allsectionsfont{\color{section_color}\fontfamily{phv}\selectfont} 
\subsectionfont{\color{subsection_color}} 

  % header and footer
\usepackage{fancyhdr}
\pagestyle{fancy}
\fancyhead{} % clear all header fields 
\fancyhead[LO,RE]{\docurl} 
\fancyhead[RO,LE]{\docver} 
\fancyfoot{} % clear all footer fields 
\fancyfoot[LE,RO]{\thepage} 
\fancyfoot[LO,RE]{\authorinfo}
\fancyfoot[CO,CE]{\today}
\renewcommand{\headrulewidth}{0.4pt}
\renewcommand{\footrulewidth}{0.4pt}


\def\mymaketitle{
\thispagestyle{plain}
\begin{center}
\vspace*{1.1cm}
\color{title_color}
\fontfamily{phv}\selectfont\LARGE{\titleinfo}\\[0.4cm]
\color{black}
\large{\authorinfo}\\[0.3cm]
\normalsize \authoremail
\end{center}
\vspace*{-0.5cm}
}

%fancy boxes
\usepackage[tikz]{bclogo}
\usepackage[blur]{pstricks}

% information boxes
\newenvironment{infobox}{%
\bigskip
\renewcommand\logowidth{40pt}
\begin{bclogo}[logo=\bcinfo, couleur = LightSteelBlue1, epBarre = 4, couleurBarre = DeepSkyBlue3,ombre = true, couleurOmbre = Snow3,blur, noborder = true, marge=8, arrondi = 0.4,epOmbre = 0.18]{Information}%
}
{%
\renewcommand\logowidth{17pt}
\end{bclogo}%
}%

% warning boxes
\newenvironment{warnbox}{%
\renewcommand\logowidth{40pt}
\bigskip
\begin{bclogo}[logo=\bcattention, couleur = LightPink1, epBarre = 4, couleurBarre = Red1,ombre = true, couleurOmbre = Snow3,blur, noborder = true, marge=8,arrondi = 0.4, epOmbre = 0.15]{Warning}%
}
{%
\renewcommand\logowidth{17pt}
\end{bclogo}%
}%


%table stuff
\usepackage{rotating}
\usepackage{booktabs}
\usepackage{multirow}
%% more space between the caption and the object
% important for the tables, they look ugly with the caption just sitting on top
\setlength{\abovecaptionskip}{10pt}   % 0.5cm as an example
\setlength{\belowcaptionskip}{10pt}   % 0.5cm as an example

\usepackage{subcaption} % multiple figures each with its own caption
%easily make the text 10^x
\providecommand{\e}[1]{\ensuremath{\!\times\! 10^{#1}}}

%missing:
% for code use package istings

% to generate filler text
\usepackage{lipsum} 

%to better include code
\usepackage{listings}
 
\definecolor{dkgreen}{rgb}{0,0.6,0}
\definecolor{gray}{rgb}{0.5,0.5,0.5}
\definecolor{mauve}{rgb}{0.58,0,0.82}
 
\lstset{ %
  basicstyle=\footnotesize,           % the size of the fonts that are used for the code
  numbers=left,                   % where to put the line-numbers
  numberstyle=\tiny\color{black},  % the style that is used for the line-numbers
  stepnumber=1,                   % the step between two line-numbers. If it's 1, each line 
                                  % will be numbered
  numbersep=5pt,                  % how far the line-numbers are from the code
  backgroundcolor=\color{white},      % choose the background color. You must add \usepackage{color}
  showspaces=false,               % show spaces adding particular underscores
  showstringspaces=false,         % underline spaces within strings
  showtabs=false,                 % show tabs within strings adding particular underscores
  frame=single,                   % adds a frame around the code
  rulecolor=\color{black},        % if not set, the frame-color may be changed on line-breaks within not-black text (e.g. commens (green here))
  tabsize=2,                      % sets default tabsize to 2 spaces
  captionpos=b,                   % sets the caption-position to bottom
  breaklines=true,                % sets automatic line breaking
  breakatwhitespace=false,        % sets if automatic breaks should only happen at whitespace
  title=\lstname,                   % show the filename of files included with \lstinputlisting;
                                  % also try caption instead of title
  keywordstyle=\color{blue},          % keyword style
  commentstyle=\color{dkgreen},       % comment style
  stringstyle=\color{mauve},         % string literal style
  escapeinside={\%*}{*)},            % if you want to add a comment within your code
  morekeywords={*,...}               % if you want to add more keywords to the set
}

%%%%%%%%%%%%%%%%%%%%%%%%%%%%%%%%%%%%%%%%%%%%%
% EDIT bellow: author and document info
%%%%%%%%%%%%%%%%%%%%%%%%%%%%%%%%%%%%%%%%%%%%%
%fill the tile
\newcommand{\titleinfo}{Documentation template in LaTeX}
%fill the author
\newcommand{\authorinfo}{Pedro  Parracho}
\newcommand{\authoremail}{ pedro.parracho@gmail.com} 
%fill documentation location
\newcommand{\docurl}{} 
 %update the version
 \newcommand{\docver}{v1}
 \newcommand{\docsubjectr}{Document Template by Pedro Parracho}
 
  %%%%%%%%%%%%%%%%%%%%%%%%%%%%%%%%%%%%%%%%%%%%
 
 % pdf stuff
\usepackage[
pdftex,colorlinks={false},
urlcolor={black},
pdfauthor={\authorinfo},
pdftitle={\titleinfo},
pdfsubject={\docsubjectr},
pdffitwindow={false},
pdfstartview={FitH},% Fit the page horizontal
citebordercolor={white}, % no border in the cite links
linkbordercolor={white},% no border in the links
urlbordercolor={white}% no border in the url links
]{hyperref}

%title
\title{\titleinfo}
\author{\authorinfo \\ \authoremail }
\date{}                                           % Activate to display a given date or no date


%%%%%%%%%%%%%%%%%%%%%%%%%%%%%%%%%%%%%%%%%%%%
%%%%%%%%%%%%%%%%%%%%%%%%%%%%%%%%%%%%%%%%%%%%
%%%%%%%%%%%%%%%%%%%%%%%%%%%%%%%%%%%%%%%%%%%%
\begin{document}

\mymaketitle
\thispagestyle{fancy}

\begin{abstract}
This is a template I wrote in order to have something to produce, nice,  small documents to in document stuff, mainly code.\\
With that goal, I went through a bunch of packages, and leave here some settings I liked to use in my documents.
The source of this document can also be useful for you to learn some tricks about how some of these packages.
In this document I use \emph{fullpage} for smaller margins, \emph{xcolor} to define nice colors, \emph{abstract} to change abstract settings, \emph{fancyhdr} for the headers, \emph{sectsty} to redefine section title font, \emph{bclogo} and \emph{pstricks} for nice colorful boxes,  \emph{rotating},  \emph{booktabs} and   \emph{multirow} for tables, \emph{subcaption} for images side-by-side and, \emph{listings} to include code in the document.
\end{abstract}

%%%
\section{Defined boxes}
Sometimes you want to call out the attention of the reader for some important aspect or comment, with that in mind I defined two  types of text boxes, the info box and the warning box, that can be used with simple environments. 


Here is the example of a info box: 

\begin{lstlisting}[language={[LaTeX]TeX}]
\begin{infobox} 
this is the text that goes inside a info box $\phi$ \lipsum[1]
\end{infobox}
\end{lstlisting}

That will look like this:

\begin{infobox} 
this is the text that goes inside a info box $\phi$ \lipsum[1]
\end{infobox}


And here is the example of a warning box:
\begin{lstlisting}[language={[LaTeX]TeX}]
\begin{warnbox} 
 this is the text that goes inside a warning box $\Phi$  \lipsum[1]
\end{warnbox}
\end{lstlisting}

\begin{warnbox} 
 this is the text that goes inside a warning box $\Phi$  \lipsum[1]
\end{warnbox}

%%%
\section{Images}
Images are a important part of documentation, we must not forget them, as one image can be worth thousand words.

\subsection{Single images}

\begin{figure}[htp]
\begin{center}
\includegraphics[width=.3\textwidth]{figs/fig1}
\end{center}
\caption[Inclision of a PNG]{Example of the inclusion of a PNG}
\label{pngfig}
\end{figure}

Another example would be how to include a jpg in a document... This can be achieved by:

\begin{lstlisting}[language={[LaTeX]TeX}]
\begin{figure}[htp]
\begin{center}
\includegraphics[width=.3\textwidth]{figs/fig2}
%% This line defines the with of the figure
\end{center}
\caption[Inclision of a JPEG]{Example of the inclusion of a JPEG}
\label{jpgfig}
\end{figure}
\end{lstlisting}

And the result can be seen Fig. \ref{jpgfig}.

\begin{figure}[htp]
\begin{center}
\includegraphics[width=.3\textwidth]{figs/fig2}
\end{center}
\caption[Inclision of a JPEG]{Example of the inclusion of a JPEG}
\label{jpgfig}
\end{figure}


\subsection{Multiple Images}

Here \ref{fig:side} we can see the 2 figures side by side.

\begin{figure}[ht]
\centering
\resizebox{0.48\textwidth}{!}{%
\includegraphics*{figs/fig1}}
\hspace{\fill}
\resizebox{0.48\textwidth}{!}{%
\includegraphics*{figs/fig2}}
\caption{An Example on how to place 2 pictures side by side.}
\label{fig:side}
\end{figure}

 Now using the subcaption package that will allow for several captions, one under each figure. When can even refer to the left on using \ref{fig:PNG}.

\begin{figure}[ht]
\centering
        \begin{subfigure}[b]{0.4\textwidth}
                \centering
                \includegraphics[width=\textwidth]{figs/fig1}
                \caption{Image of a PNG}
                \label{fig:PNG}
        \end{subfigure}%
        ~ %add desired spacing between images, e. g. ~, \quad, \qquad etc. 
          %(or a blank line to force the subfigure onto a new line)
        \begin{subfigure}[b]{0.4\textwidth}
                \centering
                \includegraphics[width=\textwidth]{figs/fig2}
                \caption{Image of a JPEG}
                \label{fig:Jpeg}
        \end{subfigure}
        \caption{Two images side-by-side, with each having its caption}\label{fig:imagesside}
\end{figure}

%%%
\section{Code}
Here you will find how to add some code to the document
the package listings is  a good way to include code.

You can include the whole file:
\lstinputlisting[language=Python,caption={Full Python file included in the text},label={cod:python}]{code/Python1.py}
or just a snipet:
\begin{lstlisting}[language={C++},caption={Snippet of C++ code  included in the text},label={cod:cpp}]
/*
* Function to print a vector to given stream
*/
template<typename T>
void PrintVector(std::ostream& ostr, const std::vector<T>& t, 
                              const std::string& delimiter = " ,")
{
    // This will print the vector
    std::copy(t.begin(), t.end(), 
                     std::ostream_iterator<T>(ostr, delimiter.c_str()));
}
\end{lstlisting}

%%%
\section{References used in the document}

This I will place and example how to use references in \LaTeX, a very nice book to start is the wikibook of \LaTeX on the web \cite{bib:wikibook}. Don't forget that after adding references you will need to compile twice, for the references to show up correctly.

%%%
\section{Tables}
Here I'm going to spread around a few tables, just to get a feel for it.
\subsection{Horizontal}
Here you find a nice horizontal table \ref{tab:ThisIsAHorizontalTable} in page \pageref{tab:ThisIsAHorizontalTable}.

\begin{table}
	\caption{This is a horizontal table}
	\label{tab:ThisIsAHorizontalTable}
	\centering
      \begin{tabular}{l r @{.} l r @{.} l r @{.} l r @{.} l r @{.} l}
         \toprule
                      &    \multicolumn{ 10}{c}{Activity  $[$Bq$]$ } \\
         \cmidrule(l){2-11}
                      &    \multicolumn{2}{c}{15 days} &   \multicolumn{2}{c}{3 months} &   \multicolumn{2}{c}{6 months} &      \multicolumn{2}{c}{1 year} &    \multicolumn{2}{c}{2 years} \\
         \midrule
         $^{37}$Ar    &   $1$&$02\times 10^{10}$ &  $2$&$22\times 10^{9}$ &   $3$&$66\times 10^{8}$ &   $9$&$95\times 10^{6}$&   $7$&$22\times 10^{3}$ \\                                                                                                                                 
         $^{39}$Ar    &   $1$&$24\times 10^{7 }$ &  $1$&$24\times 10^{7}$ &   $1$&$24\times 10^{7}$ &   $1$&$24\times 10^{7}$&   $1$&$24\times 10^{7}$ \\                                                                                                                                  
         $^{22}$Na    &   $3$&$26\times 10^{7 }$ &  $3$&$09\times 10^{7}$ &   $2$&$89\times 10^{7}$ &   $2$&$53\times 10^{7}$&   $1$&$94\times 10^{7}$ \\                                                                                                                                  
         $^{35}$S     &   $5$&$65\times 10^{8 }$ &  $3$&$07\times 10^{8}$ &   $1$&$49\times 10^{8}$ &   $3$&$53\times 10^{7}$&   $1$&$96\times 10^{6}$ \\                                                                                                                                  
         $^{33}$P     &   $7$&$63\times 10^{8 }$ &  $9$&$25\times 10^{7}$ &   $7$&$64\times 10^{6}$ &   $5$&$22\times 10^{4}$&   $2$&$37$ \\                                                                                                                                   
         $^{32}$P     &   $1$&$25\times 10^{9 }$ &  $3$&$16\times 10^{7}$ &   $1$&$98\times 10^{6}$ &   $1$&$61\times 10^{6}$&   $1$&$60\times 10^{6}$ \\                                                                                                                                
         $^{32}$Si    &   $1$&$62\times 10^{6 }$ &  $1$&$62\times 10^{6}$ &   $1$&$61\times 10^{6}$ &   $1$&$61\times 10^{6}$&   $1$&$60\times 10^{6}$ \\                                                                                                                                  
         $^{3}$H      &   $6$&$08\times 10^{8 }$ &  $6$&$01\times 10^{8}$ &   $5$&$93\times 10^{8}$ &   $5$&$76\times 10^{8}$&   $5$&$45\times 10^{8}$ \\                                                                                                                                   
         $^{7}$Be     &   $2$&$16\times 10^{8 }$ &  $7$&$88\times 10^{7}$ &   $2$&$40\times 10^{7}$ &   $2$&$22\times 10^{6}$&   $1$&$87\times 10^{4}$ \\                                                                                                                                   
         $^{207}$Bi   &   $1$&$66\times 10^{5 }$ &  $1$&$66\times 10^{5}$ &   $1$&$65\times 10^{5}$ &   $1$&$63\times 10^{5}$&   $1$&$59\times 10^{5}$ \\                                                                                                                                   
         $^{241}$Am   &   $5$&$00\times 10^{3 }$ &  $5$&$00\times 10^{3}$ &   $5$&$00\times 10^{3}$ &   $4$&$99\times 10^{3}$&   $4$&$98\times 10^{3}$ \\
         \bottomrule
      \end{tabular}
\end{table}          

\subsection{Vertical}
%%%%%%%%%%%%%
Were you can find how to place vertically a table that is too wide to fit horizontaly, see \ref{tab:ThisIsAVerticalTable} in page \pageref{tab:ThisIsAVerticalTable}.
\begin{sidewaystable}
%\begin{table}
   \caption{This is a vertical table}
	\label{tab:ThisIsAVerticalTable}
   \centering    
   \begin{tabular}{r c c c c c c}
      \toprule
                &         15 days &         1 month &        3 months &        6 months &          1 year &         2 years \\
      \midrule
Effective dose to Infants$[$Sv$]$ & \multirow{ 2}*{$7.82\times 10 ^{-7}$} & \multirow{2}*{$7.42\times 10 ^{-7}$} & \multirow{2}*{$6.71\times 10 ^{-7}$} & \multirow{2}*{$6.19\times 10 ^{-7}$} & \multirow{2}*{$5.39\times 10 ^{-7}$} & \multirow{2}*{$4.14\times 10 ^{-7}$} \\
Ext. exposure \& Inhalation & \multirow{2}*{} & \multirow{2}*{} & \multirow{2}*{} & \multirow{2}*{} & \multirow{2}*{} & \multirow{2}*{} \\
\midrule
Effective dose to Adults $[$Sv$]$ & \multirow{2}*{$7.77\times 10 ^{-7}$} & \multirow{2}*{$7.40\times 10 ^{-7}$} & \multirow{2}*{$6.70\times 10 ^{-7}$} & \multirow{2}*{$6.17\times 10 ^{-7}$} & \multirow{2}*{$5.38\times 10 ^{-7}$} & \multirow{2}*{$4.14\times 10 ^{-7}$} \\

Ext. exposure \& Inhalation & \multirow{2}*{} & \multirow{2}*{} & \multirow{2}*{} & \multirow{2}*{} & \multirow{2}*{} & \multirow{2}*{} \\
\midrule
Effective dose to Infants $[$Sv$]$ & \multirow{2}*{$2.79\times 10 ^{-4}$} & \multirow{2}*{$1.55\times 10 ^{-4}$} & \multirow{2}*{$3.70\times 10 ^{-5}$} & \multirow{2}*{$2.56\times 10 ^{-5}$} & \multirow{2}*{$2.07\times 10 ^{-5}$} & \multirow{2}*{$1.55\times 10 ^{-5}$} \\

Ingestion, summer release & \multirow{2}*{} & \multirow{2}*{} & \multirow{2}*{} & \multirow{2}*{} & \multirow{2}*{} & \multirow{2}*{} \\
\midrule
Effective dose to Adults $[$Sv$]$ & \multirow{2}*{$5.03\times 10 ^{-5}$} & \multirow{2}*{$2.98\times 10 ^{-5}$} & \multirow{2}*{$9.86\times 10 ^{-6}$} & \multirow{2}*{$7.46\times 10 ^{-6}$} & \multirow{2}*{$6.07\times 10 ^{-6}$} & \multirow{2}*{$4.56\times 10 ^{-6}$} \\

Ingestion, summer release & \multirow{2}*{} & \multirow{2}*{} & \multirow{2}*{} & \multirow{2}*{} & \multirow{2}*{} & \multirow{2}*{} \\
\midrule
Effective dose to Infants $[$Sv$]$ & \multirow{2}*{$4.95\times 10 ^{-7}$} & \multirow{2}*{$4.75\times 10 ^{-7}$} & \multirow{2}*{$4.19\times 10 ^{-7}$} & \multirow{2}*{$3.67\times 10 ^{-7}$} & \multirow{2}*{$3.05\times 10 ^{-7}$} & \multirow{2}*{$2.30\times 10 ^{-7}$} \\

Ingestion, winter release & \multirow{2}*{} & \multirow{2}*{} & \multirow{2}*{} & \multirow{2}*{} & \multirow{2}*{} & \multirow{2}*{} \\
\midrule
Effective dose to Adults $[$Sv$]$ & \multirow{2}*{$4.95\times 10 ^{-7}$} & \multirow{2}*{$4.75\times 10 ^{-7}$} & \multirow{2}*{$4.19\times 10 ^{-7}$} & \multirow{2}*{$3.67\times 10 ^{-7}$} & \multirow{2}*{$3.05\times 10 ^{-7}$} & \multirow{2}*{$2.30\times 10 ^{-7}$} \\

Ingestion. winter release & \multirow{2}*{} & \multirow{2}*{} & \multirow{2}*{} & \multirow{2}*{} & \multirow{2}*{} & \multirow{2}*{} \\
\bottomrule
\end{tabular}  
%\end{table}
\end{sidewaystable}


\begin{thebibliography}{9} %up to 10 references

\bibitem{bib:wikibook}  \emph{Wikibook on \LaTeX }, \url{http://en.wikibooks.org/wiki/LaTeX}


\end{thebibliography}

\end{document}  